\begin{entry}[2]{PCA - Part 1: Isometries }

\begin{entrysection}
In the next few entries I will write about Principal Component Analysis (PCA) and Singular Value Decomposition (SVD).
\end{entrysection}

\begin{entrysection}[Isometries (a.k.a Unitary operators)]
Let's start with a definition of what is an isometry on a real finite dimensional space $V$. An operator $S \in \L(V)$ is called an isometry if 
\[ \norm{Sv} = \norm{v} \]
for all $v \in V$. In other words, an operator is an isometry if it preserves norms (Axler). One can easily prove that this is equivalent to:
\[ SS^{\intercal} = S^{\intercal}S = I. \]
So what does this mean? The first equation implies that if such a transformation is applied to a rigid body, it won't be distorted or deformed, as it will keep its relative distances the same. This means that rotations are isometries for example.\\

\noindent Isometries are commonly used to change the basis of a vector to a new orthonormal basis. For example, let's say we are in $\R^n$, and we want to change the basis to another orthonormal basis, say 
$f_1, \ldots, f_n.$
Thus, the behaviour of our matrix, let's name it $M$, should be such that:
\[ M(v) =  \begin{bmatrix}\inp{v, f_1}\\\inp{v, f_2}\\\vdots\\\inp{v, f_n} \end{bmatrix}.\]
From the definition of matrix multiplication, it can be easily seen that a matrix whose rows are the new basis will do the trick:
\[ M(v) = \begin{bmatrix} f_1^\intercal\\ f_2^\intercal\\\vdots\\ f_n^\intercal \end{bmatrix}v = \begin{bmatrix}f_1^\intercal v\\f_2^\intercal v\\\vdots\\f_n^\intercal v \end{bmatrix} = 
\begin{bmatrix}\inp{v, f_1}\\\inp{v, f_2}\\\vdots\\\inp{v, f_n} \end{bmatrix}.\] 
In fact, for any basis of any subspace $U$ of $V$, a matrix whose rows are the basis of that subspace defines a projection matrix from $V$ unto $U$. But in the case in which $U = V$ and the basis of $U$ is orthonormal, that matrix is an isometry. What's cool about isometries though is that we can easily revert back to the original basis by applying the transpose of the matrix $M^{\intercal}$, as $M^{\intercal} = M^{-1}$.
\end{entrysection}

\end{entry}
