\begin{entry}[2]{PCA - Part 1: Isometries }

\begin{entrysection}
In the next few entries I will write about Principal Component Analysis (PCA) and Singular Value Decomposition (SVD).
\end{entrysection}

\begin{entrysection}[Isometries (a.k.a Unitary operators)]
Let's start with a definition of what is an isometry on a real finite dimensional space $V$. An operator $S \in \L(V)$ is called an isometry if 
\[ \norm{Sv} = \norm{v} \]
for all $v \in V$. In other words, an operator is an isometry if it preserves norms (Axler). One can prove that this is equivalent to:
\[ SS^{\intercal} = S^{\intercal}S = I. \]
So what does this mean? The first equation implies that if such a transformation is applied to a rigid body, it won't be distorted or deformed, as it will keep its relative distances the same. This means that rotations are isometries for example.\\

\noindent Isometries are commonly used to change the basis of a vector to a new orthonormal basis. For example, let's say we are in $\R^n$, and we want to change the basis to another orthonormal basis, say 
$f_1, \ldots, f_n.$
Thus, the behaviour of the matrix, let's name it $M$, that takes us back from the new basis to the standard basis should be such that:
\[ M\begin{bmatrix}1 \\ 0 \\ \vdots \\ 0\end{bmatrix} = f_1,  \ldots,  M\begin{bmatrix}0 \\ \vdots \\ 0\\ 1\end{bmatrix} = f_n. \]
It becomes clear then that the columns of this matrix is the new basis
\[ M = [f_1 \quad \ldots \quad f_n]\] 
The matrix's inverse $M^{-1}$ that takes us from the standard basis to the new basis should be such that:
\[ M^{-1}(f_1) = \begin{bmatrix}1 \\ 0 \\ \vdots \\ 0\end{bmatrix}, \ldots,  M^{-1}(f_n) = \begin{bmatrix}0 \\ \vdots \\ 0\\ 1\end{bmatrix}. \]
Which can be clearly seen as taking the projection unto the new basis:
\[ M^{-1}(v) =  \begin{bmatrix}\inp{v, f_1}\\\inp{v, f_2}\\\vdots\\\inp{v, f_n} \end{bmatrix}.\]
From the definition of matrix multiplication, it becomes evident that a matrix whose rows are the new basis will do the trick:
\[  M^{\intercal}(v) = \begin{bmatrix} f_1^\intercal\\ f_2^\intercal\\\vdots\\ f_n^\intercal \end{bmatrix}v = \begin{bmatrix}f_1^\intercal v\\f_2^\intercal v\\\vdots\\f_n^\intercal v \end{bmatrix} = 
\begin{bmatrix}\inp{v, f_1}\\\inp{v, f_2}\\\vdots\\\inp{v, f_n} \end{bmatrix} = M^{-1}(v).\] 
Which means that $M$ is an isometry.
\end{entrysection}

\end{entry}
